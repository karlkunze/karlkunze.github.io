% Options for packages loaded elsewhere
\PassOptionsToPackage{unicode}{hyperref}
\PassOptionsToPackage{hyphens}{url}
\PassOptionsToPackage{dvipsnames,svgnames*,x11names*}{xcolor}
%
\documentclass[
  11pt,
]{article}

\usepackage{lmodern}
\usepackage{amssymb,amsmath}
\usepackage{ifxetex,ifluatex}
\ifnum 0\ifxetex 1\fi\ifluatex 1\fi=0 % if pdftex
  \usepackage[T1]{fontenc}
  \usepackage[utf8]{inputenc}
  \usepackage{textcomp} % provide euro and other symbols
\else % if luatex or xetex
  \usepackage{unicode-math}
  \defaultfontfeatures{Scale=MatchLowercase}
  \defaultfontfeatures[\rmfamily]{Ligatures=TeX,Scale=1}
  \setmainfont[]{cochineal}
  \setsansfont[]{Fira Sans}
  \hyphenchar\font=-1
\fi

% Use upquote if available, for straight quotes in verbatim environments
\IfFileExists{upquote.sty}{\usepackage{upquote}}{}
\IfFileExists{microtype.sty}{% use microtype if available
  \usepackage[]{microtype}
  \UseMicrotypeSet[protrusion]{basicmath} % disable protrusion for tt fonts
}{}
\makeatletter
\@ifundefined{KOMAClassName}{% if non-KOMA class
  \IfFileExists{parskip.sty}{%
    \usepackage{parskip}
  }{% else
    \setlength{\parindent}{0pt}
    \setlength{\parskip}{6pt plus 2pt minus 1pt}}
}{% if KOMA class
  \KOMAoptions{parskip=half}}
\makeatother
\usepackage{xcolor}
\IfFileExists{xurl.sty}{\usepackage{xurl}}{} % add URL line breaks if available
\IfFileExists{bookmark.sty}{\usepackage{bookmark}}{\usepackage{hyperref}}
\hypersetup{
  pdftitle={Résumé},
  pdfauthor={Karl Kunze},
  colorlinks=true,
  linkcolor=Maroon,
  filecolor=Maroon,
  citecolor=Blue,
  urlcolor=blue,
  pdfcreator={LaTeX via pandoc}}
\urlstyle{same} % disable monospaced font for URLs
\usepackage[top=.5in, left =.5in, right=.5in, bottom=.75in]{geometry}
\setlength{\emergencystretch}{3em} % prevent overfull lines
\providecommand{\tightlist}{%
  \setlength{\itemsep}{0pt}\setlength{\parskip}{0pt}}
\setcounter{secnumdepth}{-\maxdimen} % remove section numbering

\title{Résumé}
\usepackage{authblk}
              
            \author{Karl Kunze}
            \date{02/20/2023}


%% should be top-aligned in case of uneven vertical length)
\newenvironment{columns}[1][]{}{}
%%
\newenvironment{column}[1]{\begin{minipage}[t]{#1}\ignorespaces}{%
\end{minipage}
\ifhmode\unskip\fi
\aftergroup\useignorespacesandallpars}
%%
\def\useignorespacesandallpars#1\ignorespaces\fi{%
#1\fi\ignorespacesandallpars}
%%
\makeatletter
\def\ignorespacesandallpars{%
  \@ifnextchar\par
    {\expandafter\ignorespacesandallpars\@gobble}%
    {}%
}
\makeatother

% Use fontawesome. Note: you'll need TeXLive 2015. Update.
\usepackage{fontawesome}

% Mess with sections
\usepackage{titlesec}
\usepackage{sectsty}
% \sectionfont{\rmfamily\mdseries\large\bf\underline}
\sectionfont{\normalfont\sffamily\large\bfseries\sectionrule{0pt}{0pt}{-4pt}{1pt}}
\subsectionfont{\rmfamily\mdseries\normalsize\scshape}
\titlespacing\section{0pt}{12pt plus 4pt minus 2pt}{4pt plus 2pt minus 2pt}
\titlespacing\subsection{0pt}{12pt plus 4pt minus 2pt}{4pt plus 2pt minus 2pt}

\usepackage{enumitem}
\setlist[itemize]{leftmargin=*}



% Make AP style (kinda) dates for the updated/today field

\usepackage{datetime}
\newdateformat{apstylekinda}{%
  \shortmonthname[\THEMONTH]. \THEDAY, \THEYEAR}

% Fancyhdr, as I tend to do with these personal documents.
\usepackage{fancyhdr,lastpage}
\pagestyle{fancy}
\renewcommand{\headrulewidth}{0.0pt}
\renewcommand{\footrulewidth}{0.0pt}
\lhead{}
\chead{}
\rhead{}
\lfoot{}
\cfoot{\scriptsize  Karl Kunze - Résumé - \emph{Updated:} \apstylekinda\today }
\rfoot{\scriptsize \thepage/{\hypersetup{linkcolor=black}\pageref{LastPage}}}



% Always load hyperref last.
\usepackage{hyperref}
\PassOptionsToPackage{usenames,dvipsnames}{color} % color is loaded by hyperref

\hypersetup{unicode=true,
            pdftitle={Karl Kunze  (R\'{e}sum\'{e})},
            pdfauthor={Karl Kunze},
            colorlinks=true,
            linkcolor=Maroon,
            citecolor=Blue,
            urlcolor=blue,
            breaklinks=true, bookmarks=true}

\begin{document}
% shift=(current page.north east)
%\begin{wrapfigure}{r}{\textwidth}

 % includephoto
%\end{wrapfigure}

\flushleft{\huge \bfseries Karl Kunze}

\flushleft{\footnotesize

\faEnvelopeO \hspace{1 mm} \href{mailto:}{\tt \href{mailto:khk44@cornell.edu}{\nolinkurl{khk44@cornell.edu}}} \hspace{1 mm}
\faPhone \hspace{1 mm}  \hspace{1 mm}
\faMapMarker \hspace{1 mm} 422 Bradfield Hall, Ithaca NY,
14853 \hspace{1 mm}
}
\vspace{-1em}
\flushleft{\footnotesize
\faGlobe \hspace{1 mm} \href{http://karlkunze.github.io}{\tt karlkunze.github.io}  \hspace{1 mm} 
\faTwitter \hspace{1 mm} \href{http://twitter.com/@kunzx37}{\tt @kunzx37} \hspace{1 mm}
\faLinkedin \hspace{1 mm} \href{https://www.linkedin.com/in/karlkunze}{\tt karlkunze} \hspace{1 mm}
\faGithub \hspace{1 mm} \href{http://github.com/karlkunze}{\tt karlkunze} \hspace{1 mm}
}

\begin{column}{0.44\textwidth}

\hypertarget{education}{%
\section{Education}\label{education}}

\hypertarget{undergraduate}{%
\subsection{Undergraduate}\label{undergraduate}}

\begin{itemize}
\tightlist
\item
  B.S. Plant Science concentration in Plant Breeding and Genetics, minor
  in Business for Life Sciences- Cornell University \hfill 2013-2017
\end{itemize}

\hypertarget{graduate}{%
\subsection{Graduate}\label{graduate}}

\begin{itemize}
\tightlist
\item
  PhD Candidate, advisor Dr.~Mark Sorrells, Cornell University Graduate
  School, Field of Plant Breeding and Genetics, minor in Plant Pathology
  and Food Science \hfill 2017-present
\end{itemize}

\hypertarget{leadership-activities-and-professional-services}{%
\section{Leadership Activities and Professional
Services}\label{leadership-activities-and-professional-services}}

\begin{itemize}
\item
  Student representative of the Crop Science Society of America(CSSA)
  executive board \hfill 2022-present
\item
  Representative of the CSSA science policy committee
  \hfill 2022-present
\item
  Local Graduate Student Liaison member for the National Association of
  Plant Breeders(NAPB) Graduate Student Working Group \hfill Fall
  2020-August 2021
\item
  Cornell Plant Breeding and Genetics Graduate Student
  Association(Synapsis)-President \hfill 2018-2019
\item
  Synapsis Professional Development Committee Member \hfill 2020-2021
\item
  Corteva Symposium organizing committee member~ \hfill     April 2019
\item
  Graduate Student Representative on the Cornell Plant Breeding Faculty
  Search Committee \hfill Spring 2019
\end{itemize}

\hypertarget{awards-and-grants-received}{%
\section{Awards and Grants Received}\label{awards-and-grants-received}}

\begin{itemize}
\item
  Graduate student of USDA OREI grants program 2017-51300-26809 and
  2020-51300-32179 ``Developing Multi-use Naked Barley for Organic
  Systems approx. \$3,000,000''
\item
  ``Genetic Characterization of Germination Traits and Their
  Relationship to Preharvest Sprouting in Winter and Spring Barley award
  of \$11,500''- American Malting barley Association Grant Award
  \hfill July 2021-June 2022
\item
  Cornell Plant Breeding and Genetics Munger-Murphy Award \hfill August
  2022
\item
  Gerald O. Mott Award Recipient \hfill March 2022
\item
  Recipient of the ASA, CSSA, SSSA Future Leaders in Science Award
  \hfill December 2018
\end{itemize}

\end{column}

\begin{column}{0.02\textwidth}
~

\end{column}

\begin{column}{0.48\textwidth}

\hypertarget{experience}{%
\section{Experience}\label{experience}}

\hypertarget{research-projects}{%
\subsection{Research Projects}\label{research-projects}}

\begin{itemize}
\item
  Measure components of weed competitive ability in organic naked barley
  variety trials by using field trait phenotypes and aerial imaging to
  measure barley vigor and growth over 4 trials and 2 years
\item
  Genome wide association studies of organic barley diversity panels
  across 13 location by year field locations throughout the Northern
  United States using 50K barley SNP markers
\item
  Genetic by environmental analysis of winter naked barley variety
  trials across 8 environments throughout the Northern United States
\item
  Evaluation of dormancy and pre-harvest sprouting across a winter
  malting barley breeding population consisting of 450 lines over two
  location and two years.
\item
  Evaluation of malting quality at the USDA ARS Cereal Crops Research
  Unit in Madison, WI ~ \hfill December 2021 and January-February 2022
\end{itemize}

\hypertarget{technical-skills}{%
\section{Technical Skills}\label{technical-skills}}

\begin{itemize}
\item
  Highly proficient in operating and data collection of a plant breeding
  program including measurement field phenotypes and integrating genetic
  data to make selection decision and publish original research
\item
  Highly experienced in processing, analysis, organization, and
  experimental design of field trials of a novel barley breeding program
\item
  Highly proficient in R statistical software and Excel for data
  management and analysis. Moderate proficiency in using git version
  control
\item
  Basic proficiency in Unix shell scripting and command line. Limited
  experience with Python and Docker.
\item
  Highly proficient in flying unmanned aerial systems for imaging of
  plant variety trials and breeding populations \& Certified FAA UAS
  Part 107 Remote Pilots License. \hfill    2019-present
\item
  Basic proficiency in Agisoft Pro and Open Drone Map software stitching
  applications.
\item
  Basic proficiency related to chemistry for malting quality analysis
\item
  Demonstrated ability to work in multi-institutional collaborative
  projects
\item
  Participated in an International exchange workshop between Cornell
  University, Tokyo University of Agriculture and Technology and the
  Technical University of Munich \hfill October 2019
\end{itemize}

\end{column}

\newpage

\begin{column}{0.44\textwidth}

\hypertarget{professional-societies}{%
\section{Professional Societies}\label{professional-societies}}

\begin{itemize}
\item
  Crop Science Society of America(CSSA) student member
  \hfill 2019-present
\item
  National Association of Plant Breeders (NAPB) student member
  \hfill 2020-present
\item
  New York State Agriculture Society member \hfill 2017-present
\item
  Cornell Plant Breeding and Genetics Graduate Student Association,
  Synapsis \hfill 2017-present
\item
  Alumni of Alpha Gamma Rho, Professional and Social Agricultural
  Fraternity, Zeta Chapter
\end{itemize}

\hypertarget{publications}{%
\section{Publications}\label{publications}}

\begin{enumerate}
\def\labelenumi{\arabic{enumi}.}
\item
  Massman, C., Meints, B., Hernandez, J., Kunze, K., Smith, K. P.,
  Sorrells, M. E., \ldots{} \& Gutierrez, L. Crop Science(2023) Genomic
  prediction of threshability in naked barley.
  \url{https://doi.org/10.1002/csc2.20907}
\item
  Travis E. Rooney, Karl H. Kunze, Mark E. Sorrells. The Plant
  Genome(2022) Genome wide marker effect heterogeneity is associated
  with a large effect dormancy locus in winter malting barley.
  \url{https://doi.org/10.1002/tpg2.20247}
\item
  Bunting, J. S., Ross, A. S., Meints, B. M., Hayes, P. M., Kunze, K.,\&
  Sorrells, M. E. (2022). Effect of Genotype and Environment on
  Food-Related Traits of Organic Winter Naked Barleys. \emph{Foods},
  \emph{11}(17),2642.\url{https://doi.org/10.3390/foods11172642}
\item
  Chris Massman, Brigid Meints, Javier Hernandez, Karl Kunze, Patrick
  M.Hayes, Mark E. Sorrells, Kevin P. Smith, Julie C. Dawson, and Lucia
  Gutierrez. Crop Science(2022) Genetic Characterization of Agronomic
  Traits and Grain Threshability for Organic Naked Barley in the
  Northern U.S. \url{https://doi.org/10.1002/csc2.20686}
\item
  Sweeney, D.W., Kunze, K.H. \& Sorrells, M.E. QTL x environment
  modeling of malting barley preharvest sprouting. Theor Appl Genet
  (2021). \url{https://doi.org/10.1007/s00122-021-03961-5}
\end{enumerate}

\end{column}

\begin{column}{0.02\textwidth}
~

\end{column}

\begin{column}{0.50\textwidth}

\hypertarget{outreach}{%
\section{Outreach}\label{outreach}}

\begin{itemize}
\item
  Interviewed with the Craft Maltsters guild on
  \href{https://craftmalting.com/field-to-bench-consumers-and-collaboration-spur-progress-in-barley-breeding-at-cornell-university/}{Breeding
  Malting Barley for New York State} \hfill February 2022
\item
  Presented on winter malting barley breeding progress for New York at
  the New York State Empire malting barley summit at the Culinary
  Institute of America, Hyde Park, NY \hfill December 2022
\item
  Co-led a weekly graduate student journal club with Will Stafstrom.
  Topics were related to current research and topics in the fields of
  plant breeding, genetics and crop science. \hfill Spring 2022
\item
  Spoke at numerous annual field days to discuss barley breeding and
  organic naked barley to the general community
\item
  Wrote a brief article titled growing malting barley amid climate
  change for the
  \href{https://ambainc.org/news-details.php?id=63d014dfba04a}{American
  Malting Barley Association} \hfill       September 2022
\item
  Guest on the \href{https://www.youtube.com/watch?v=Dw_8N39wyBI}{``All
  Things Agriculture Podcast''} with Eric Carey
\item
  Presented an eOrganic webinar titled
  \href{https://eorganic.info/node/23566}{``Progress on Organic Naked
  Barley Breeding, Exploration of Organic Breeding Traits''}
  \hfill April 2021
\end{itemize}

\hypertarget{research-and-conference-presentations}{%
\section{Research and Conference
presentations}\label{research-and-conference-presentations}}

\begin{itemize}
\item
  Presented current research and status of the NY winter malting barley
  project at the NYS Empire Malt and Barley Summit at the Culinary
  Institute of America, Poughkeepsie, NY \hfill December 2022
\item
  Presented research of ``Interaction of Pre-harvest sprouting,
  germination rate and malting quality for winter and facultative
  malting barley'' and ``Genotype by Environment Interaction of Organic
  Winter Naked Barley'' at CSSA annual meeting in Baltimore, Maryland ~
  \hfill November 2022
\item
  Presented research of ``Interaction of Pre-harvest sprouting,
  germination rate and malting quality for winter and facultative
  malting barley'' at the 23rd North American Barley Researchers
  Workshop and 43rd Barley Improvement Conference UC Davis,CA ~
  \hfill September 2022
\item
  Presented research of ``Developing Winter Malting Barley for New York
  State'' Michigan Beer and Malt Conference in Traverse City, MI ~
  \hfill January 2022
\item
  Research Presentation titled ``Components of Weed Competitive
  Ability'' at CSSA,ASA and SSSA Tri-societies annual meeting in Salt
  Lake City, Utah ~ \hfill November 2021
\item
  Presenter at the Philly Malt and Grains Conference, Virtual
  \hfill March 2021
\end{itemize}

\end{column}

\end{document}
